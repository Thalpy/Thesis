\section{Abstract}

This thesis presents a comprehensive study of the interplay of forces at the nanoscale in colloidal systems, utilizing Atomic Force Microscopy (AFM) as the primary investigative tool. The research explores a solution of 50:50 water glycerol solution across a range of different LiCl salt concentrations. Detailed examinations of surface imaging and force curve analysis are conducted, highlighting the intricacies of colloidal interactions and their practical implications for collodial physics. Through a series of methodical AFM experiments, the work delves into the operational parameters and their implications, culminating in a profound analysis of approach and retract force curves. The study not only contributes to the fundamental understanding of colloidal dynamics but also bridges the gap between theoretical knowledge and practical applications, paving the way for innovations in materials science, biotechnology, and environmental sustainability.

\section{Lay Abstract}

In this thesis, we explore the tiny forces that act on microscopic particles suspended in liquids, known as colloids. These particles, which are often too small to see with the naked eye, play a critical role in everyday products like paints, foods, and medicines. By using a sophisticated tool called Atomic Force Microscopy, we can ``feel" the forces as these tiny particles come close to or move away from each other. Our research shows how these forces change from pushing the particles apart to pulling them together, and how this delicate balance affects the behavior of colloids in different environments. Understanding these forces helps us not only to appreciate the complexities of microscopic world but also to develop better materials and technologies for everyday use, from smarter drugs to more sustainable practices in industries.