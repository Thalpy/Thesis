\section{Introduction to the Thesis}

This thesis delves into the complex world of colloidal interactions, focusing on the microscale forces that govern the behavior of particles in various environments. The thesis begins with a curiosity about the forces affecting colloidal particles and extends into a comprehensive exploration of their impacts in real-world scenarios, bridging theoretical knowledge with practical experiments.

\subsubsection{Chapter 1: Introduction to Surface Forces}

The first chapter sets the stage by examining the fundamental interactions between particles, exploring the theoretical understand of colloidal stability. Initially various forms of explaining colloidal behaviour is discussed, such as electrostatic repulsion and van der Waals attraction. This culminates in the defining theory for colloidal physics; Boris Derjaguin and Lev Landau, Evert Verwey and Theodoor Overbeek theory (DLVO theory). 

An example is given of DLVO in action, and explores how these interactions influence the stability and behavior of colloidal suspensions, contributing to advancements in materials science and biotechnology.

\subsubsection{Chapter 2: Atomic Force Microscopy}
This chapter delves into the history and development of microscopy, leading up to the modern techniques of Atomic Force Microscopy (AFM). It covers the basic operational principles of AFM, including tip calibration, common artifacts, and imaging modes. This chapter is essential for understanding the detailed methodologies used in subsequent experimental analyses as the majority of the data generated in the thesis is from AFM.

\subsubsection{Chapter 3: Surface Imaging}
In this chapter, the focus shifts to the practical application of AFM in surface imaging. Initially an example with mica is used to demonstrate some of the methods and results of imaging AFM. Borosilica surfaces are explored using surface analysis techniques, discussing the resolution of silica particle surfaces. This chapter lays the groundwork for understanding the surface roughnesses of the probing colloidal particle at the surface level, as well as providing a surface roughness analysis of the borosilica glass surface used, which is crucial for the later analysis of force curves.

\subsubsection{Chapter 4: Atomic Force Spectroscopy Analysis}
This chapter describes the experimental setup and methods used for force curve collection in AFM studies. It explains the computational methods for interpreting these curves, ensuring a repeatable method across a consistent experimental setup. The chapter is critical in the detailed examination of approach force curves for the following chapters and lays the foundation of the script used, as well as the intensive validation methods used. Said script was designed to process high volumes of data automatically, and is provided on Github as FOSS software. It also overviews the two AFMs used in the force curve analysis. 

\subsubsection{Chapter 5: Analysis of Approach Force Curves in Colloidal Systems}
Here, the thesis focuses providing a clear method for extracting the force at contact from raw data in colloidal systems. The chapter focuses on presenting and justifying the contact forces calculated for each concentration site. For each site, a snapshot of the analytical curves is shown to highlight where the data points are derived from, justifying the averaged force curve. Additionally, the range of contact forces is calculated, giving an averaged approach force with standard deviation for each site and concentration combination. Finally the chapter is finished with an analysis of the ionic strength vs approach force analysis.

\subsubsection{Chapter 6: Analysis of Retract Force Curves in Colloidal Systems}
Following a similar trajectory to chapter 5, this chapter focuses on the retract curves, Once again justifying the individual fits and providing an averaged force curve and retract force. This retract force is then analysed in a similar manner at the end of the chapter.

\subsubsection{Chapter 7: Operational Parameter Exploration and Their Implications}
This chapter explores various operational parameters in AFM studies and their implications. It includes analysis of tip speed, dwell time, pH, and force mapping. The chapter builds upon the foundational data presented earlier to further understand and interpret the results of AFM force curve analysis under different conditions to highlight potential differences and implications..

\subsubsection{Chapter 8: Further Analysis and Discussion}
The final chapter analyzes and discusses the findings presented in the previous chapters, focusing on the details and implications of AFM force curve analysis. This chapter also delves into the interesting quirks seen in the dataset, such as the ``shelf" feature present in some of the curves. Further reflection is also done on the implications and success of the open source script and the impact of investigating a range of operation parameters. Additionally, the data generated is compared against rheological data, as well as DLVO itself to draw further conclusions and implications.