%%%%%%%%%
% TO DO %
%%%%%%%%%
% Expand intro
% Redo graphs from graphing program, expand scope, involve 
% More detail
% Actually describe motivation again
% Rewrite into based off previous chapters


%Don't write too much
%Write what is needed in the text
%The more you write, the higher probability of making mistakes - keep to the point
%Describe results
%Explanation after desc
%Have to be phrases that make sense/say something
%Have a purpose

\chapter{Analysis of Approach Force Curves in Colloidal Systems}
\label{chap:approach_force_curves}

This chapter aims to serve a simple goal. To provide a clear route to force at contact extraction from raw data. As the raw unprocessed data is severely voluminous the scope of this chapter is kept to one goal, provide the data and justification for the contact forces calculated for each concentration, site and parameter. The results of this analysis is then concluded in Chapter 7, which, free of the need to justify every data point and thus focuses on the conclusions from this data. 

\section{MFP-1D contact force derivation}

For the data collected from the MFP-1D the analytical approach to this elucidation is given in Chapter 4. This section will overview the results for the following LiCl concentrations: 0.6mM, 1.6mM, 5mM, 10mM, 25mM, 50mM, 230mM, 550mM. Then the impact of pH, piezo movement (Hz), differing tips and contact force will be concluded.

\subsection{Rejected curves}

 During the process, several curves were rejected from the data set as part of the processing method. Due to the sheer volume of curves (10,000+ curves) involved in this analysis.  This was either due to machine failure to engage in the surface, or due to significant noise due to mechanical failure (Over-correction in peizo movement by the AFM or vibrations in the building.) While the initial curves taken on the machine were too noisy, further repeats and refinements to the procedure eventually resulted in usable curves. In some of the data collected, the impact from residual vibrations was mitigated in later datasets by operating the AFM in the night. Other improvements were found by optimising the AFM parameters  - such as tip speed, dwell time (contact and retract) and target force. The optimal parameters found were 0.5Hz for tip speed, including a short dwell time after the curve was taken and a target force of 8-12nN. 

\begin{figure}
    \centering
    \includegraphics[width=0.75\linewidth]{chapter5/miss_error.png}
    \caption{A graph demonstrating a rejected curve. In this case the AFM failed to reach the surface for the recorded data, leading to a single up and down motion. This was rejected due to }
    \label{fig:miss-error}
\end{figure}

The other reason for rejected curves was due to the shift of the snapshot window providing too little data in either the approach phase or the contact phase (covered in Chapter 4). 

\subsection{Contact force calculations}

In this results section, the contact force graphs displayed represent only a subset of the total analyzed graphs, selected to illustrate the most significant aspects of the analysis procedure. The remainder of the curves, not shown here, played an instrumental role in guiding the parameter optimization process for the data processing script. This process involved the review and refinement of several thousand curves in order to deal with the noise intrinsic to the system. Ensuring a good fit for these curves is paramount; otherwise, one risks obtaining erratic and misleading graphs that could compromise the integrity of the data interpretation. The selection process for the resulting curves involved the exclusion of certain outliers and repeat measurements, a necessary step taken to refine the data and enhance the clarity of the observed trends. Three types of graphs were chosen for their analysis of the fitting parameters and results: the histogram of contact forces provides a statistical view of the interaction forces at the point of contact, demonstrating the range of forces across the graph. The log-linear plot of the force as a function of the Z-piezo position (in nanometers), highlighting of the separation distance between the AFM tip and the sample surface during the interaction phase. At the top of the graph the transition to the contact phase can be seen. This point of transition is the contact force. The logarithmic scale for force highlights the sensitivity of the AFM in detecting forces at the nanoscale. Finally the overall approach force curve derived from binned data presents a view of the resulting averaged force curve behavior during the approach phase. Each graph was selected for its ability to support the chosen parameters and thus the resulting contact force.


%0.6mM
\insertapproachfigures{0.6}{2}{Site one demonstrates one of the difficulties with collecting the data at the lower ranges of the molar concentrations. As the repulsive force is over a wider range there is a smaller window captured of the contact phase. This means that when trying to fit to the data, there is less data available to provide a solid fit.  As such, a slight bowing effect is seen in the binned average curve.}
\insertapproachfigures{0.6}{3}{Site two demonstrates an ideally fitted curve, however in the log-linear plot an unusually is seen - the point in which the contact force area is raised up the graph. This is due to the large amount of noise seen during this interface phase, giving a large variation in the contact force histogram.}
% ... and so on for other sites and concentrations.

%1.6


\section{Effects of hydrodynamics}

\newpage.
\newpage

\section{Dwell time effects}

\newpage.
\newpage.
\newpage

\section{pH effects}

\newpage
\newpage

\section{Force mapping}