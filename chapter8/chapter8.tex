\chapter{Bacterial colloids AFM}

\section{Introduction}

In addition to this in liquid force curves were produced from probing bacteria. The possibility of force mapping a bacterial surface was investigated, but ultimately ruled impossible on the current machine. However efforts have been made to procure access to a more suitable AFM (See section 2.1).

\subsection{AFM tip treatment} \chapter{Force analysis}

This chapter aims to analyse the findings presented in the previous two chapters, focusing on the intricate details and implications of Atomic Force Microscopy (AFM) force curve analysis. The preceding chapters laid a foundational understanding of the colloidal interactions and the work involved to process the raw data into a usable format. This chapter instead will focus on using the previous chapters' data to draw conclusions about collodial systems. The force-distance profiles obtained from AFM are analysed, providing a direct and nuanced insight into the particle interactions at an individual level. The force curves, reflecting the precise nature of forces acting at the nanoscale, will be juxtaposed against the bulk property measurements to draw a more comprehensive picture of the colloidal dynamics. This analysis will not only validate but also potentially challenge and refine the conclusions drawn from the bulk measurements, offering a more holistic understanding of the colloidal systems under study. Through this integrative approach, we aim to bridge the gap between our microscopic observations and derive behaviour on the macro scale. This chapter takes the following format, firstly over viewing the approach data, then the retract data.

\section{Approach force curves}



The interaction forces in macroscale colloidal suspensions was calculated as $F∗$, derived from rheological data and zeta potential measurements. This parameter provides a context between macroscopic rheological behavior and microscopic inter-particle forces in the context of varying ionic strengths. The calculation of $F∗=2.5d2σ∗$ was based on the Derjaguin approximation, a theoretical framework interaction forces in systems where particles are significantly larger than the range of their interaction forces, as overviewed in chapter 1. The choice of $F∗$ as a reference for comparison against the $F0$ values calculated by the previous 2 chapters, this was driven by the desire to establish a correlation between bulk suspension properties and individual particle interactions. By varying the ionic strength, we observed significant changes in $F∗$, reflecting alterations in the colloidal interactions. These changes were then meticulously compared with the $F0$ values to validate our approach and to provide a understanding of the colloidal behavior under different electrostatic conditions. This comparative analysis not only reinforced the validity of using $F∗$ as a proxy for understanding colloidal forces but also underscored the interplay between bulk rheological properties and particle-level forces.

\begin{figure}[!tbph!!!]
  \centering
  \subfloat[Site 1 curves.]{\includegraphics[width=0.45\textwidth]{chapter5/ContF1.png}\label{fig:CF1}}
  \hfill
  \subfloat[Site 2 curves.]{\includegraphics[width=0.45\textwidth]{chapter5/ContF2.png}\label{fig:CF2}}
  \caption{The resultant calculated repulsive force between the surface and the tip. The AFM data is compared against data taken from rheology. In the case of 550mM the repulsive forces transition to attractive, corresponding to the onset stress transitioning into yield stress (as seen from rheological data). F0 is taken from when the curve is within the onset of the contact region, Fc is taken from when the curve bends towards the surface, at the onset of contact (observable in \textit{Fig.3}. (a) and (b) are site 1 and 2 respectively\cite{John}.}
\end{figure}

\section{Shelf analysis}

The shelf seen in some datapoints 

\section{Variability in data}

The variability (noise) in the data may be due to the change in force applied to the surface. This is because the AFM is old and doesn't apply consistent pressure. It may have an affect, but you'd expect to see it more pronouced in the different salt concentrations. Plot the std dev vs concentration graph.

There is *alot* of noise at this scale. Because 
- I think it's cause it's a short period between phases?
- It's an old machine
- It hates me

Talk about how much noise there is. There is a lot of noise.

Show graphs

You can't just Lmao bin more cause then you have no data




Successful adaption and development of the glass treatment to AFM tips was performed. In particular a tip was treated with DCDMS surface coating. This tip was planned to be used with force mapping to investigate the presence of adhesive patches theorized by the group \cite{Teun1} and other literature \cite{Patchy}. 

\subsection{Attaching bacteria to a cantilever}
\newpage

\section{Bacterial force curves}

\newpage
\newpage
\newpage

\chapter{Conclusion}