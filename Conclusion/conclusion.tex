\chapter{Conclusion}

The culmination of this experimental research into the nanoscale interface of colloidal interactions, explored through the lens of Atomic Force Microscopy (AFM), presents a range of findings that both challenge and enrich our current understanding. This thesis has delved deep into the complexities of force-distance profiles, revealing the nuanced and intricate nature of particle interactions in colloidal systems. These interactions, as observed, are not only varied across different surface sites but also exhibit a remarkable sensitivity to environmental conditions, such as ionic strength and concentration of LiCl. This variability and sensitivity highlight the inherently dynamic character of colloidal systems, which has been a focal point of this study.

A detailed analysis of AFM force-distance profiles was provided, highlighting the nuanced nature of particle interactions at the nanoscale. Both approach and retract curves were analysed in a novel method of averaging several hundred curves per site. This provided a robust analysis of each site, allowing for the noise profile normally seen in AFMs to be tamed. This noise profile was mitigated to the point of extracting features otherwise unobservable (i.e. the ``shelf"). The results also indicated significant variability in interaction forces across different surface sites. This variability could stem from surface heterogeneities, variations in local chemical composition, and structural differences. 

From this analysis changes in ionic strength was seen to influence colloidal interactions. As concentrations increased this led to decreased repulsive forces in the approach curves, with the higher concentrations flipping to attractive forces. For retract curves, higher concentrations were seen to increase the adhesive force of the particle. For both the approach and retrace, concentrations past 50mM seemed to be a tipping point for these effects. This finding underscores the impact of ionic effects on colloidal behavior and interaction forces, and potentially provides validation for the charge screening effect theory.

The dynamic nature of colloidal interactions was highlighted from research into tip speed and dwell time analysis. The influence of tip speed highlighted that during slower speeds the AFM tip can potentially cause the surface to restructure (such as ion redistributing) in response to the gradual introduction of the tip surface. The influence of dwell time on attractive forces between the AFM tip and surface suggests a time-dependent response in the colloidal system, possibly due to ion reorganization or surface conditioning over time. 

The consistent presence of a shelf feature across various concentrations and its correlation with specific attractive forces pointed towards complex, non-DLVO interactions at play. This shelf tied into previous observations in literature \cite{Kilpatrick2013DirectlyProbing}, with our results partially validating their findings. The origin of this shelf could be a result of factors like ion-ion correlations, surface charge heterogeneities, hydration forces, or other specific ion effects.

The experimental results were validated against classical DLVO theory, revealing limitations of DLVO in explaining certain observed phenomena like the 'shelf' feature in force curves. When surface roughness was accounted for the fit with DLVO improved, though still had issues. DLVO also doesn't account for the shelf features seen at all, indicating a need for refinement in theoretical models. 

Extending to the macroscopic realm, the implications of this study was considered for colloidal rheology, particularly in the context of shear thickening behavior in dense colloidal suspensions. When compared against empirical evidence \cite{reference4} our data supports the theory that frictional contact is a pivotal factor in shear thickening, thus bridging the gap between nanoscale interactions and macroscopic phenomena.

In addressing the practical challenges encountered in this research, the development of software tools for enhanced AFM data analysis stands out as an achievement. These open source tools not only facilitate a more robust analysis of complex data sets, particularly from older AFM systems, but also democratize access to high-quality research opportunities by extending the functionality of existing equipment.

In summary, this thesis has not only contributed to a deeper understanding of the intricacies of colloidal interactions at the nanoscale but has also laid the groundwork for further exploration. By highlighting these areas of interest, it paves the way for a more comprehensive and accurate representation of the complex world of colloidal science. Additionally, the software provided allows other scientists to use the tools created in their own research.

\section{Future work}

This thesis has highlighted several potential avenues of further research that could be further expanded upon.

The impact of ionic strength and concentration on colloidal interactions, particularly the transition from repulsive to attractive forces, offers an interesting area of exploration. This could involve systematic experiments across a wider range of ionic conditions to map out the thresholds and dynamics of these transitions.

Two points of evidence exist to imply that surface roughness has a significant effect on colloidal interaction on small length scales: The rheological analysis and the comparison against DLVO theory. Future research could delve deeper into the role of surface roughness, both from a rheological perspective or from better application of theoretical DLVO surface roughness models. This might involve more detailed surface characterization and modeling to understand how surface topography at the nanoscale influences colloidal behavior.

The shelf highlighted that hydration forces could play a role in the colloidal interactions observed. Future work could focus on explicitly investigating these forces, or expressly focus on investigating this observed shelf in further detail, possibly through experimental setups designed to isolate and measure hydration effects in different colloidal systems.

The influence of factors like tip speed and dwell time on force measurements indicates a dynamic and time-dependent nature of colloidal systems. Further research could explore these temporal aspects more deeply, possibly integrating rheological measurements to connect nanoscale interactions with macroscopic properties. As rheology is familiar with hysteric effects, probing the origin of these properties on the interface can benefit both the nanoscale and macroscale.

Given the limitations of existing theories like DLVO in explaining certain phenomena observed in our data, a suitable area for further work is the refinement of these theoretical models by a theoretician. This could involve incorporating factors such as surface roughness, ion-ion correlations, and charge heterogeneities into the models. Revision of DLVO to incorporate the attractive shelf force seen in the dataset is an additional avenue open to exploration. Developing new theoretical frameworks or computational models that better capture the complex realities of colloidal interactions observed would be a significant contribution to the field.

The potential connection between nanoscale colloidal interactions and macroscopic phenomena like shear thickening in dense suspensions presents an exciting area for future research. This could involve correlating AFM measurements with bulk rheological properties to better understand the underlying mechanisms of shear thickening and other colloidal phenomena.

The software tools developed for enhanced AFM data analysis represent a significant area of interest too. Future work could involve applying these tools to a broader range of datasets, enhancing their capabilities, and possibly integrating them with other analytical methods to provide a more comprehensive analysis for a range of AFM force curves. Given that the code is available on a open source repository, there is a large potential for a range of contributors to help combat the use of proprietary software in AFM space.

In summary, this thesis, like many thesis before it, has only seemed to open up more avenues of explorations and ask more questions than it solves itself. This author, in particular, is excited to see where the future lies as a result of this investigation.
