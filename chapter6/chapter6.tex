\chapter{Retract force curves}

\section{Introduction}

\subsubsection{Analysis of Silica-silica retract curves}

Further analysis is currently being performed on the retract curves utilizing an adapted version of the approach script. This is done in a similar method to the previous, with parameters refined by eye, then the range tested by adding or subtracting 10nm to the contact fitting region. The literature surrounding silica-silica force curves taken in salt usually omit the retract curves, while only a few mention them \cite{Retrace}. 

\cite{John}.

\newpage

\insertretractfigures{6}{0.6}{2}{Site one demonstrates one of the difficulties with collecting the data at the lower ranges of the molar concentrations. As the repulsive force is over a wider range there is a smaller window captured of the contact phase. This means that when trying to fit to the data, there is less data available to provide a solid fit.  As such, a slight bowing effect is seen in the binned average curve.}
\insertretractfigures{6}{0.6}{3}{Site two demonstrates an ideally fitted curve, however in the log-linear plot an unusually is seen - the point in which the contact force area is raised up the graph. This is due to the large amount of noise seen during this interface phase, giving a large variation in the contact force histogram.}
% ... and so on for other sites and concentrations.

%1.6
\insertretractfigures{6}{1.6}{1}{Site one represents an idealised curve. Where the processing is able to extract out a clear signal from the noise, and thus a clearly distributed contact force.}
\insertretractfigures{6}{1.6}{2}{Site 2 demonstrates an interesting feature - a region where there seems to be two contact phases. As DLVO doesn't explain any dual barrier features this is rather unexpected. One explanation may be that as the two surfaces approach one another some aspect (for example sterics) prevents full contact between the sphere and surface, which then slips or is overcome, allowing full contact later. It is important to note that this feature is present throughout the dataset, and survived the binning and averaging process, and thus is a consistent feature in this site.}

%5
\insertretractfigures{6}{5}{1}{Site one demonstrates a similar, but weaker feature seen in 1.6mM site 2 observable in the log-lin plot. The overall binned fit demonstrates a suitable fit, so it remains to be a feature of this site.}
\insertretractfigures{6}{5}{2}{Site two demonstrates the same feature, but slightly less prominent again.}
\insertretractfigures{6}{5}{3}{Site three once again has the feature seen across all of the force curves at this concentration}

%10 2 sites
\insertretractfigures{6}{10}{1}{}
\insertretractfigures{6}{10}{2}{}

%25 3 sites
\insertretractfigures{6}{25}{1}{}
\insertretractfigures{6}{25}{2}{}
\insertretractfigures{6}{25}{3}{}

%50 2 sites
\insertretractfigures{6}{50}{1}{}
\insertretractfigures{6}{50}{2}{}

%230 2 sites
\insertretractfigures{6}{230}{1}{}
\insertretractfigures{6}{230}{2}{}

%550 3 sites
\insertsnowflakefigures{5}{550}{1}{}
\insertsnowflakefigures{5}{550}{2}{}
\insertsnowflakefigures{5}{550}{3}{}

\subsection{0.6mM}
%This is saved locally
\newpage.
\newpage

\subsection{1.6mM}
.
\newpage.
\newpage

\subsection{5mM}
.
\newpage.
\newpage

\subsection{10mM}

\newpage

\subsection{25mM}

\newpage

\subsection{50mM}

\newpage

\subsection{230mM}
.
\newpage.
\newpage

\subsection{550mM}
.
\newpage.
\newpage


