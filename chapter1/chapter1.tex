\chapter{Introduction to surface forces}
%Intro
%Surface forces - van der waals, electrostatics, charge screening, DLVO, derjaguin
%hydrodynamics - lubriciating, viscous forces
%brownian 
%Rheology - bulk & translation, shear thickening, friction, industry
%MOVE OUR SYSTEM TO CHAPT4
%Conclision
%Check what I define what a long range force is, what is a short range force is?
%add nomenclature/glossary
%Where is Ue(r) defined ?
%REFERENCES ARE BUGGERED PLEASE FIX BEFORE SENDING OFF TO BE READ!!!
 
\title{Introduction}
%The methods of how particles interact with one another has been a longstanding area of interest in physics. As matter encroaches upon one another %there in an interplay of attractive and repulsive forces.
The fundamental interactions between particles has long been an area of active research in physics. The interplay of attractive and repulsive forces which occurs at the boundary between particles is both difficult to understand and vital to master.
%These forces and mechanisms are known as interacting surface forces, and the interplay of these interactions define the systems we experience and observe in day to day life. Some examples include the wetting of surfaces from an interfacing liquid or the means in which an organism adheres to a surface. 
Examples of the emergent behaviour which arises from these interactions includes the spreading of a water droplet across a surface and the adherance of micro-organisms to surfaces.
These forces
%while from the perspective of the universe are nothing new,
have
%gained
recieved increasing interest and understanding over the last century
%to
from scientists and industry alike. The demand for a clear and concise picture of these interactions only increases as innovations in technology rely more and more on these interactions, from touch screen developments, 
%water resistant
superhydrophobic surfaces, bacterial frustrating surfaces or even the humble custard. 
There are several ways to 
%flirt with
probe these forces 
%either as an innovator; 
either by chemically treating the surface, altering the physical structure of said surface or by altering the
%inter-facial 
nature of the 
%solute 
solution said surface finds itself involved with. By changing these properties, new promising technologies can 
%push
advance our understanding of such 
%a
systems.
%to it's limits. 
As %scientific study
researchers become
%s
more and more involved on 
%elucidating
studying these interactions,
%on the molecular scale
 we find more and more of the world we live in is 
%defined by our understanding 
a result of the emergent behaviour of such systems and as a result, it is imperative to not only question the history of our combined understanding, but to test such systems experimentally.
%The history of surface forces finds itself
Our modern understanding of surface forces is derived from van der Waals 
%forces, 
theory,
%a theory produced by 
the result of a series of papers produced by London, Debye and Keesom \cite{AFMvdW}. %These 
Attractive forces were combined with repulsive double layer forces to %form 
give the main theoretical basis %of 
for particle-particle interactions, %consequently 
subsequently %known 
referred to as DLVO theory (derived from Boris Derjaguin and Lev Landau, Evert Verwey and Theodoor Overbeek) as a result of two groups reaching the same conclusion independently\cite{Verwey}\cite{DERJAGUIN}. This underlying theory describes the %resultant 
surface forces felt between two particles in solution .
%From these simple particle-particle interactions bulk behaviours are derived, 
More complex behaviours arise emerge as a result of these interactions - for example %such as the case for 
colloidal systems. Mixed phase suspensions, called colloids are a combination of a solvent and a solute. Indeed, one only has to look inwards to find a fascinating colloidal system (albeit a very complicated one). \cite{surfThesis} \cite{christian2018a}
Systems defined by inter-molecular forces are prevalent in a wide range of applications, said theory providing the basis for several industries such as water purification, batch processing (food, pharmaceuticals, detergents, paints) and mining.\cite{TABOR19772}
%Born from (history)
\section{Forces between molecules}
%wrong - don't trust interpretation 
Inter-molecular forces are essentially electrostatic in nature, as postulated from quantum theory, initially defined by the Hellman-Feynman theorem. %From the Coulomb force and complex fluctuating forces observed around the surfaces of atoms. 
However, %simple solutions to the Schrödinger are hard to come by, 
solving Schrodinger's equation for poly-atomic systems is very difficult and as such, this unified force is fractured down into smaller %classifications 
components to simplify their definition and %equation
computation. These categories are %known as 
ionic forces, van der Waals forces, hydrophobic forces, hydrogen bonding and solvation forces. 
%Covalent / coloumb interactions, short range and strong
Chemical bonding depends on the interaction of neighboring atomic orbitals, while steric forces arise from quantum mechanical or electrostatic interactions between separated particles. %They are an emergent property of the electronic structure of the atoms, as apposed to a molecular orbitals, where the molecular configuration is in a state of semi-perminance and free from flux changes.
%A main difference between the two is the permanence of charge distribution changes, where in the case of the former is retained as long as the bond is retained.
Coulomb's %forces 
law %are 
describes the force %of the charge effects of two charges applied upon one another with respect to distance. 
which arises from the electrostatic interaction of charged particles:
%loose
%In  approximation the inverse square force is given by: %Not atoms per say
%Why is this the simpliest
\begin{equation}
F \propto \frac{Q_1 Q_2}{kr^2}
\end{equation}
Where upon this can be expanded into:
\begin{equation} %\frac{Q_1 Q_2}{4\pi\varepsilon_0\varepsilon r^2} =
F = -\frac{dw(r)}{dr} =  \frac{z_1z_2e^2}{4\pi\varepsilon_0\varepsilon r^2}
\end{equation}
where w(r) is the free energy of the Coulomb interaction, r is the distance between the two charges Q\textsubscript{1} and Q\textsubscript{2}. $\varepsilon$ is dielectric constant of the medium, and z is the ionic valency of the atom in question, in relation to the elementary charge e. % Due to it's inverse square law nature it is a force highly dependant on distance. 
The inverse square relationship between the free energy of interaction and distance between the point charges makes this force drop off sharply as the distance increases.
%w(r) is Joules, ~ 8x10^-19J, kT is 5x10^-21J. Stronk.
%lennard jones potential here?
\section{Brownian Motion}
%Traditionally; 
The term colloid was coined in 1861, drawn from the Greek word $κόλλα$ (kolla), meaning glue, from Thomas Graham's observations of particle aggregation\cite{old_colloid}. %Nowadays 
today, a colloid is defined as %the intermixing 
a mixture of two separate phases; a dispersed, suspended phase and a continuous%, medium
phases in which the former find themselves in. %For our area of interest, we place our scope on a particular colloidal suspension; a dispersion of solid particles within a liquid medium. 
Colloids consisting of solid particles suspended in a liquid medium shall be the focus of this thesis.\cite{review_colloid1}
Consider the idea of marbles kept within a liquid. At rest, they would lie upon the bottom of the container, yielding to the force of gravity\cite{Neuton}. However, as you scale down these marbles, to smaller and smaller sizes, the kinetic energy of the system is enough for keep the marbles dispersed within the liquid, due to Brownian motion\cite{Brown}.
At micrometer sizes however, the marbles begin to affect one another; when two marbles are brought together by Brownian motion, provided the interactions between the two of them are attractive, they will attract towards each other, and eventually aggregate, until they are large enough to sediment again. If there is no attraction, or indeed repulsion, between the marbles, they will stay suspended within the solution.
%Interactions between the marble's surfaces, are defined by a few fundamental forces resolving linearly with one another. These interplay of forces, borne from electrostatics, van der Waals and solvation forces, all sum up into a force profile based on distance between the two surfaces.
The various interactions between the marble surfaces can be combined to give a force profile, describing the strength of the interaction between the marbles with respect to the distance between them.
These force profiles depend upon a number of % intrinsic 
properties intrinsic to the phenomena. %To define these variables in a broad stroke; the liquid the solid colloid is immersed in (liquid phase properties), the surface properties of the solid phase (solid phase properties) and interactions between same phase particles (same phase interacting properties).
On the macro scale, these interactions %equal out to resolve into their 
ultimatley contribute to bulk properties. As this system is disturbed by external forces, a 
%resultant 
characteristic relaxation time is observed, dependant on the ions in solution and surface properties of the solid phase particles. %From these disturbances, new dipole moments can arise additionally. 
These properties give rise to the %hysteric effects 
time dependent effects seen in systems such as these.
With such a complex system of non trivial interactions, the history of elucidating such a complex system has not been a simple one. Ideally in physics finding a unifying, complete equation to describe all manner of interactions would be the goal, however, even if that were possible, colloidal systems are not frozen in an equilibrated state and are the result of combining relations between intrinsic properties and dynamic, sympathetic effects. As such the current state of the theory relies upon simplified simulations to push forward the field, and as such, an experimental analysis %upon 
 of said models is required. \cite{FoundColloidBook}\cite{IsGreenBook}

