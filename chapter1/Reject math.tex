\section{Lennard's potato}

\begin{equation} %Basic principles of colloid science
U_{rep} = \left(\frac{B}{a}\right) e^{-ar}
\end{equation}

Where $a$ and $B$ are constants. The Lennard-Jones potential is a result of these two equations, given by:

\begin{equation} %Basic principles of colloid science
U = U_{rep} + U_{att} = \left( \frac{B'}{r^{12}} \right) - \left( \frac{A'}{r^6}\right)
\end{equation}

%rewrite
Where $A'$ is a constant, related to the intrinsic quantum mechanical properties of the molecules in question.\cite{London}:

\begin{equation} %Basic principles of colloid science
A' = \left( \frac{3}{4} \right) h v \alpha^2
\end{equation}
%


%Consider Faraday's law - but only kinda. A_h = A' \pi^2 q^2
This expression describes $A'$ for two homogeneous molecules. $\alpha$ is the polarisability of the atom or molecule, $h$ is Plank's constant (6.63x10$^{-34}$ Js) and $v$ is the required frequency of a photon in order to ionise the molecule. In brief this equation describes the relationship between the van der Waals radius and the ionisation enthalpy, and thus encapsulates the interplay from the distance vs the size of the particles itself (as electron shell size increases, less energy is needed to ionise).
% CHECK: DO I USE h ELSEWHERE? /!\

\subsection{Electro Kabpapoorwpagka}
\begin{equation} %From foundations of collodial science. Why is it different than everywhere else I see it? (though this isn't quite right for our system) THIS IS FOR A VACUUM
\kappa = \left(\frac{e^2\Sigma n_{oi} z^2_i}{\epsilon k_B T}\right)^2 
\end{equation}

%Basic principles of collodial science
\begin{equation} 
\frac{1}{\kappa} = \left(\frac{\epsilon kT}{e^2 \Sigma c_i z_i^2}\right)^\frac{1}{2}
\end{equation}

%Basic principles of collodial science
\begin{equation} 
\frac{1}{\kappa} = \left(\frac{\epsilon RT}{F^2 \Sigma c_i z_i^2}\right)^\frac{1}{2}
\end{equation}
%F is faraday constant

%Intermolecular and surface forces
\begin{equation} 
\kappa = \left(\frac{e^2 \sum_{i} p_{\infty i} z_i^2} {\epsilon \epsilon_0 k T}  \right)^\frac{1}{2} 
\end{equation}\textbf{}

 F is Faraday's constant, $z_i$ is the valence of i ions
 
 %Old eqn simplified r/h to cancel each other out
\begin{equation} %This can't be right??
U_E (r) = 4\pi\epsilon \frac{R}{2 + \frac{h}{R}}\Psi^2_{0} e^{-\kappa r}
\end{equation}

%Replaced
\begin{equation} 
U_E (r) = \frac{2\pi h \sigma^2 e^{-\kappa}}{\kappa \epsilon \epsilon_0}
\end{equation}

%Simplified
\begin{equation} %This can't be right??
U_E (r) = 4\pi\epsilon \frac{R}{2}\Psi^2_{0} \epsilon_0 e^{-\kappa h}
\end{equation}

 $\approx \epsilon \epsilon_0 \kappa \Psi_0$
 
%Poission boltzman Debbie shuckle

This model was then revised under the Gouy-Chapman model in 1910-1913, \cite{34} which pays respect to the thermodynamic nature of the system. Which relies on the Poisson-Boltzmann equation.

\begin{equation} %possibly unneeded
\delta ^2 \Psi = \frac{\delta^2 \Psi}{\delta x^2} + \frac{\delta^2 \Psi}{\delta y^2} + \frac{\delta^2 \Psi}{\delta z^2} = - \frac{p_e}{\epsilon \epsilon_0}
\end{equation}

Debye-H\"uckle approximation linearises said Poisson-Boltzmann equation. In cases where $\Psi$ is small enough to approximate out, the Debye-H\"uckle parameter is given by: %We CANT assume Psi, it's a factor in our system.

\begin{equation} 
\kappa =  \sqrt{\frac{2000 F^2}{\epsilon \epsilon_0 R T}} \sqrt{I}
\end{equation}

$\kappa^-1$, the which is the characteristic Debye length of a system. However, in the case of colloids, Psi cannot nessicarily be assumed, and is given by  %It can but also it can't great (maybe calculate psi)
%why does psi need k when you need psi for k

 From this basis the Helmholtz model was born to consider the ionic radius. This ionic radius is used to understand the finite ionic adsorption. %Maybe remove - unnecessary detail?
 
\begin{equation}
\kappa = \sqrt{\frac{\epsilon_r \epsilon_0 k_b T}{2 N_A e^2 I}}
\end{equation}

Where $\epsilon_r$ is the dielectric constant which was taken from \cite{WaterGlycerolEpR}. $\epsilon_0$ is the absolute dielectric permittivity. $k_b$ is the Boltzmann constant, $N_A$ is Avogadro's constant,  $e$ is the charge, $I$ is ionic strength and $T$ is temperature.