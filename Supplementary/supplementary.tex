\section{Approach force curves}



The interaction forces in macroscale colloidal suspensions was calculated as $F∗$, derived from rheological data and zeta potential measurements. This parameter provides a context between macroscopic rheological behavior and microscopic inter-particle forces in the context of varying ionic strengths. The calculation of $F∗=2.5d2σ∗$ was based on the Derjaguin approximation, a theoretical framework interaction forces in systems where particles are significantly larger than the range of their interaction forces, as overviewed in chapter 1. The choice of $F∗$ as a reference for comparison against the $F0$ values calculated by the previous 2 chapters, this was driven by the desire to establish a correlation between bulk suspension properties and individual particle interactions. By varying the ionic strength, we observed significant changes in $F∗$, reflecting alterations in the colloidal interactions. These changes were then meticulously compared with the $F0$ values to validate our approach and to provide a understanding of the colloidal behavior under different electrostatic conditions. This comparative analysis not only reinforced the validity of using $F∗$ as a proxy for understanding colloidal forces but also underscored the interplay between bulk rheological properties and particle-level forces.

\begin{figure}[!tbph!!!]
  \centering
  \subfloat[Site 1 curves.]{\includegraphics[width=0.45\textwidth]{chapter5/ContF1.png}\label{fig:CF1}}
  \hfill
  \subfloat[Site 2 curves.]{\includegraphics[width=0.45\textwidth]{chapter5/ContF2.png}\label{fig:CF2}}
  \caption{The resultant calculated repulsive force between the surface and the tip. The AFM data is compared against data taken from rheology. In the case of 550mM the repulsive forces transition to attractive, corresponding to the onset stress transitioning into yield stress (as seen from rheological data). F0 is taken from when the curve is within the onset of the contact region, Fc is taken from when the curve bends towards the surface, at the onset of contact (observable in \textit{Fig.3}. (a) and (b) are site 1 and 2 respectively\cite{John}.}
\end{figure}
